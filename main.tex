\documentclass[12pt]{article}
\usepackage[utf8]{inputenc}
\usepackage[english]{babel}
\usepackage[margin=2.5cm]{geometry}
\usepackage{setspace}
\usepackage{times}

\usepackage[
backend=biber,
style=apa,
citestyle=authoryear 
]{biblatex}

\usepackage{csquotes}
 
\addbibresource{abm_misinformation_lit.bib}

\title{\vspace{.5cm }
  \LARGE{If you see something, say something (to others):} \\
  \large{The effects of socially-contingent corrections on spreads of misinformation on social networks}
  }

\author{Hyunjin Song}
\date{\today}

%\doublespacing

\begin{document}

\maketitle

\section{Introduction}

There has been an explosive interest among public and academics alike on how people directionally process and maintain factually false (or at least factually dubious) information (\cite{LEWANDOWSKY_JARMC2017}), and further, how corrections to such false information is received under various scenarios (\cites{LEWANDOWSKY_JARMC2017}; \cites{thorson_2016}). Research on political misinformation and motivated reasoning generally suggests that fact-checking (or corrections) on false information has only limited effects due to inherent tendency of humans to directionally process the (politically) relevant information. However, a small line of research suggests that the effect of political fact-checking on social network may be socially contingent.

\section{Theoretical framework}






\end{document}
